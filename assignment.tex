%Not sure on the latex tags yet this will be rough outline
%Also potentially combine this with ipython notebook
\documentclass{article}
\usepackage{times}
\usepackage{hyperref}
\usepackage{mathtools}

\begin{document}
\title{Case Study: Propagation of contagions in complex networks}
\author{Cameron Poole\\
	20254381}
\today
\maketitle

\section{Viruses, diseases and the health of networks}
\paragraph{
Supposedly computers are racing towards a technological singularity were computers will have itelligence that is comparable to humans. 
Social, biological, technological and communication networks are all examples of complex networks which behave in similar ways. One example of this is spreading phenomena exist for each of the networks information, pathogens, disease, innovation all flow through the network in a similar way, Diseases pass from individual to individual just a computer virus such as a botnet or trojan might. Harvesting an individuals contacts and then attempting to infect them. Just like real world diseases some viruses are more effective at infecting and infect individuals through different pathways.

The most basic models of disease represents individuals into three compartments susceptible, infected and recovered. The spread of disease through this mode is described by a number of differential equations.
These equations are as follows\\}
\begin{equation}
	{{dS \over{dt} } = {-\beta IS}}
	 \frac{dS}{dt} = - \beta I S 
	\frac{dI}{dt} = \beta I S - \nu I 
	\end{equation}


\paragraph{
We can represent the spreading of disease in a number of different ways(Anderson, May 1992). From simple array models similar to cellular automata, graph models based upon contact of nodes to complex models based upon agent based models in which each entity possesses a number of attributes.
}

\paragraph{
Homogenous mixing 
Many social, biological, and communication systems can be properly described as complex 
networks with nodes representing individuals or organizations and edges mimicking the 
interactions among them [1-4]. For example, the neural system can be considered as a network 
consisted of neurons connecting through neural fiber [5], the Internet is a network of many 
autonomic computers connected by optic fiber or other communication media [6]. The analogous 
examples are numerous: these include power grid networks [5], social networks [7-8], collaboration 
networks [9-10], traffic networks [11], and so onn

The spread of contagions (information, viruses, something) through networks is similar in both humans and computers.
Both social networks and computer networks are examples of complex networks, attempting to estimate the spread of a contagion through them is difficult. 
Deterministic models for epidemic disease spreading were worked out in the 1980s see wikipedia article.
There are many different models and aspects to consider in disease modelling. 
Many social, biological, and communication systems can be properly described as complex 
networks with nodes representing individuals or organizations and edges mimicking the 
interactions among them [1-4]. For example, the neural system can be considered as a network 
consisted of neurons connecting through neural fiber [5], the Internet is a network of many 
autonomic computers connected by optic fiber or other communication media [6]. The analogous 
examples are numerous: these include power grid networks [5], social networks [7-8], collaboration 
networks [9-10], traffic networks [11], and so on. 
}

\section{Different compartmental models}
\paragraph{
The overwhelming majority of disease models are based on a compartmentalization of individuals or hosts according to their disease status (Kermack & McKendrick 1927; Bailey 1957; Anderson & May 1992)

Multiple models can be used to explain how virus or disease maybe spread in a network. 
An infection may spread in multiple different complex ways. One of the first models created to explain how a disease spreads through a network is a 
SIRS model}

\section{Ways in which we can represent networks and disease}
\paragraph{Epidemiology is the study of the spread of disease in humans but many of the models can be appliead to computerws}

\section{Percolation Theory}
\url{http://en.wikipedia.org/wiki/Percolation_theory }\\
\url{http://en.wikipedia.org/wiki/Contact_process_(mathematics)} \\
\url{http://en.wikipedia.org/wiki/Directed_percolation}\\
\url{http://en.wikipedia.org/wiki/Percolation_threshold}\\
\section{Questions}
\section{Links for further research}


\url{http://en.wikipedia.org/wiki/Infectious_disease  }\\
\url{http://en.wikipedia.org/wiki/Complex_contagion   }\\
\url{http://en.wikipedia.org/wiki/Complex_network     }\\
\url{http://en.wikipedia.org/wiki/Computer_network    }\\
\url{http://en.wikipedia.org/wiki/Epidemic_models_on_lattices}\\
\url{http://en.wikipedia.org/wiki/Next_generation_matrix}\\
\url{http://en.wikipedia.org/wiki/Computer_virus}\\
\url{http://en.wikipedia.org/wiki/Computer_worm}\\
\url{http://en.wikipedia.org/wiki/Compartmental_models_in_epidemiology}\\
\url{http://simpy.readthedocs.org/en/latest/simpy_intro/process_interaction.html}\\
\url{http://pythonhosted.org/ComplexNetworkSim/start.html#start}\\
\url{http://www.biomedcentral.com/content/pdf/1471-2334-10-190.pdf}\\
\url{http://www.sciencedirect.com/science/article/pii/S0377042709007341}\\
\url{http://computationallegalstudies.com/2009/11/15/programming-dynamic-models-in-python-3-outbreak-on-a-network/}\\
\url{http://computationallegalstudies.com/2009/10/11/programming-dynamic-models-in-python/}\\
\url{http://people.oregonstate.edu/~medlockj/other/python/SIR.py}\\
\url{http://wiki.deductivethinking.com/wiki/Python_Programs_for_Modelling_Infectious_Diseases_book}\\
\url{http://weblab.com.cityu.edu.hk/blog/chengjun/understanding-basic-sir-sis-epidemic-model-with-python/}
\url{https://github.com/elofgren/zombies/blob/master/Python%20Models/zombieSIR.py}\\
\end{document}
